%\documentclass[UTF8]{ctexart}
\documentclass[12pt,a4paper]{article}

\usepackage[UTF8]{ctex}
\usepackage[left=0.5in,right=0.5in,%
top=0.75in,bottom=0.75in]{geometry}
\usepackage{amsthm}
\usepackage{amsmath}

\title{\heiti 材料表征方法}
\author{\kaishu 岚岫}
\date{\today}

\begin{document}
    \maketitle
    %\tableofcontents
\begin{abstract}
    大三下学期与大四上学期陆续修读了几门关于材料表征方法的课程,此文档将陆续整理相关知识,可作为考试复习提纲和学习参考。
\end{abstract}
\section{X射线衍射分析}
\section{电子显微分析}
\section{同步辐射}
\section{中子科学}
\section{192001材料表征上课考点}
    \begin{enumerate}
        \item X射线是如何产生的?
        \begin{enumerate}
            \item 画出X射线管的结构
            \item 主要由阴极(W灯丝)和用Cu、Cr、Fe、Mo等纯金属制成的阳极靶组成
            \item 阴极通电加热,在阴阳两极之间加以直流高压(约数万伏)
            \item 阴极发射的大量电子高速飞向阳极,与阳极碰撞产生X射线
        \end{enumerate}
        \item X射线的一些基本性质
        \begin{enumerate}
            \item X射线是一种波长很短的电磁波
            \item X射线的波长范围为0.01$nm$~0.25$nm$
            \item X射线是一种横波,由交替变化的电场和磁场组成
            \item X射线具有波粒二象性,因其波长较短,其粒子性较为突出,即可以把X射线看成是一束具有一定能量的光量子流,$$E=h v=h c / \lambda$$
            \item X射线穿过不同介质时,折射系数接近于1,几乎不发生折射现象
            \item X射线肉眼不可见,但可使荧光物质发光、能使赵向地板感光、能使一些气体产生电离现象
            \item X射线的穿透能力强,能穿透对可见光不透明的材料,特别是波长在0.1$nm$以下的硬X射线
            \item X射线照射到晶体时,将产生散射、干涉、衍射等现象,与光线的绕射现象类似
            \item X射线具有破坏杀死生物组织细胞的作用
        \end{enumerate}
        \item 管电压、管电流、阳极靶原子序数对连续谱的作用
        \item 特征X射线谱的产生
        \begin{enumerate}
            \item 当X射线管压高于靶材相应的某一特征值$U_k$时,在某些特定波长位置上,将出现一系列强度很高、波长范围很窄的线状光谱,称为特征谱或标识谱。
            其波长与阳极靶材的原子序数有确定关系,故可作为靶材的标志和特征$$\sqrt{\frac{1}{\lambda}}=K_{2}(Z-\sigma)$$表明阳极靶材的原子序数越大,同一线系的特征谱波长越短。
            \item 冲向阳极的电子若具有足够能量,将内层电子击出而成为自由电子。此时,原子处于高能不稳定状态,必然自发的向稳态过渡。若$L$层电子跃迁到$K$层填补空位,原子
            由$K$激发态转为$L$激发态,能量差以X射线的形式释放。这就是特征X射线,称为$K_\alpha$射线。
            \item 由于$L$层内还有能量差别很小的亚能级,不同亚能级的电子跃迁将辐射$K_{\alpha1}$、$K_{\alpha2}$射线。
            \item $$\lambda_{K_{\alpha 1}}<\lambda_{K_{\alpha 2}}, \quad I_{K_{\alpha 1}} \approx 2 I_{K_{\alpha 2}}$$
        \end{enumerate}
        \item X射线的真吸收
        \begin{enumerate}
            \item 光电效应:当入射X射线光量子能量等于或略大于吸收体原子某壳层电子的结合能时,电子易获得能量从内层逸出,成为自由电子,成为光电子。
            这种光子击出电子的现象成为光电效应。光电效应将消耗大部分入射能量,导致吸收系数突增。
            \item 荧光效应:因光电效应处于相应的激发态的原子,将随之发生如前所述的外层电子向内层跃迁的过程,同时辐射出特征X射线,称X射线激发
            产生的特征辐射为二次特征辐射,称这种光致发光的现象为荧光效应。
            \item 俄歇效应:原子$K$层电子被击出后,$L$层一个电子跃入$K$层填补空位,而另一个$L$层电子获得能量溢出成为俄歇电子。称这种一个$K$层空位被两个$L$
            层空位取代的过程为俄歇效应。
            \item 荧光X射线与俄歇电子均为物质的化学成分信号。荧光X射线用于重元素的成分分析,俄歇电子用于表面轻元素分析。
            \item X射线穿过物质后强度会产生衰减。强度衰减主要是由于真吸收消耗于光电效应与热效应。强度衰减还有一小部分是由于X射线偏离了原来的入射方向,即散射。
        \end{enumerate}
        \item X射线的相干散射
        \begin{enumerate}
            \item 当入射X射线与受原子核束缚较紧的电子相遇,使电子在X射线在交变电场作用下发生受迫振动,向四周辐射与入射X射线波长相同的辐射
            \item 因各电子散射的X射线波长相同,有可能相互干涉,因此称相干散射,亦称经典散射
            \item 物质对X射线的散射可以认为只是电子的散射
            \item 相干散射仅占入射能量的极小部分
            \item 相干散射是X射线衍射分析的基础
        \end{enumerate}
        \item X射线的不相干散射:当X射线与自由电子或受核束缚较弱的电子碰撞时,使电子获得部分能量离开原子核而成为反冲电子,X射线能量损失,而发生波长变长的不相干散射。
        \item 六方晶系的指数换算:$$\begin{array}{c}{U=u-t, V=v-t, W=w} \\ {u=(2 U-V) / 3} \\ {v=(2 V-U) / 3} \\ {t=-(u+v)} \\ {w=W}\end{array}$$
        \item 简单点阵的面间距公式:
        \\正交晶系:$$d_{h k l}=\frac{1}{\sqrt{h^{2} / a^{2}+k^{2} / b^{2}+l^{2} / c^{2}}}$$
        \\正方晶系:$$d_{h k l}=\frac{1}{\sqrt{\left(h^{2}+k^{2}\right) / a^{2}+l^{2} / c^{2}}}$$
        \\立方晶系:$$d_{h k l}=\frac{a}{\sqrt{h^{2}+k^{2}+l^{2}}}$$
        \\六方晶系:$$d_{h k l}=\frac{1}{\sqrt{\frac{4}{3}\left(h^{2}+h k+k^{2}\right) / a^{2}+l^{2} / c^{2}}}$$
        \item 倒易点阵的性质
        \item 非晶态物质结构的主要特征
        \item 布拉格方程的应用:
        \begin{enumerate}
            \item 布拉格方程是X射线衍射分析中最重要的基础公式,能简单方便的说明衍射基本关系
            \item 用已知波长$\lambda$的X射线照射晶体,通过衍射角$2\theta$的测量计算晶体中各晶面的面间距$d$,这就是X射线结构分析
            \item 用已知面间距$d$的晶体反射样品激发的X射线,通过衍射角$2\theta$的测量计算X射线的波长$\lambda$,这就是X射线光谱分析 
        \end{enumerate}
        \item 几种点阵的结构因子计算结果
        \begin{enumerate}
                \item 简单点阵能产生衍射的干涉面指数$(HKL)$的平方和之比为:1:2:3:4:5:\dots
                \item 体心点阵
                \begin{enumerate}
                    \item 当$H+K+L$为奇数时,$\left|\boldsymbol{F}_{H K L}\right|^{2}=\mathbf{0}$,衍射强度为零
                    \item 当$H+K+L$为偶数时,$\left|\boldsymbol{F}_{\boldsymbol{H K L}}\right|^{2}=4 f^{2}$,晶面能产生衍射,这些干涉面的指数平方和之比
                    为2:4:6:8:\dots
                \end{enumerate}
                \item 面心点阵
                \begin{enumerate}
                    \item 当$H,K,L$奇偶混合时,$\left|\boldsymbol{F}_{H K L}\right|^{2}=\mathbf{0}$,衍射强度为零
                    \item 当$H,K,L$全奇全偶时,$\left|\boldsymbol{F}_{\boldsymbol{H K L}}\right|^{2}=16 f^{2}$,晶面能产生衍射,这些干涉面的指数平方和之比
                    为3:4:8:11:12:\dots
                \end{enumerate}
        \end{enumerate}
        \item 多重因子:某种晶面的等同晶面数增加,参与衍射的几率随之增大,相应衍射强度也随之增加。晶面的等同晶面数对衍射强度的影响,称为多重因子$P$,
        多重因子与晶体的对称性及晶面指数有关。各晶系晶面族的多重因子可见下表:
        \item 非晶态物质结构的主要特征
        \item 识别PDF
        \item 光学显微镜的分辨率:$\Delta r_{0} \approx \frac{1}{2} \lambda$
        \item 在200$kV$的加速电压下,电子波的波长为0.00251$nm$,随加速电压上升,电子波波长下降,分辨率上升
        \item 电磁透镜的像差
        \begin{enumerate}
            \item 几何像差:又称为单色光引起的像差
            \begin{enumerate}
                \item 球差:由于透镜中心区域和边缘区域对电子的折射能力不同形成的。球差大小为$$\Delta r_{s}=\frac{1}{4} C_{s} \alpha^{3}$$
                公式中$C_{S}$是球差系数,$\alpha$为孔径半角,从公式可以看出,减小球差的途径是减小球差系数与小孔径角成像。
                \item 像散:由于透镜磁场非旋转对称性引起不同方向的聚焦能力出现差别。像散大小为$$\Delta r_{A}=\Delta f_{A} \alpha$$式中$\Delta f_{A}$
                为磁场出现非旋转对称时的焦距差,故可通过引入强度和方位均可调节的矫正磁场消除像散
            \end{enumerate}
            \item 色散:波长不同的多色光引起的像差。是透镜对能量不同电子的聚焦能力的差别引起的。色差大小为
            $\boldsymbol{\Delta} \boldsymbol{r}_{c}=\boldsymbol{C}_{c} \alpha\left|\frac{\boldsymbol{\Delta} \boldsymbol{E}}{\boldsymbol{E}}\right|$
            式中$C_{s}$是色差系数,$\Delta E / E$是电子能量变化率,可通过稳定加速电压和单色器来减小色差
            \item 球差系数与色差系数是电磁透镜的指标之一,其大小除了与透镜结构、极靴形状和加工精度等有关外,还受激磁电流的影响。球差系数与色差系数均随透镜激磁电
            流的增大而减小。故若要减少电磁透镜的像差,透镜线圈应尽可能通以大的激磁电流
        \end{enumerate}
        \item 分辨率:电磁透镜的分辨率由衍射效应和球面像差决定
        \begin{enumerate}
            \item 衍射效应对分辨率的影响:$$\Delta r_{0}=\frac{0.61 \lambda}{N \sin \alpha}$$
            式中$\lambda$是波长,$N$是介质的相对折射率,$\alpha$是透镜的孔径半角,
            波长$\lambda$越小,孔径半角$\alpha$越大,衍射效应限定的分辨率半径就越小,透镜的分辨率就越高
            \item 由球差、像散、色差所限定的分辨率分别为$\Delta r_{S}$、$\Delta r_{A}$、$\Delta r_{C}$,其中球差是限制透镜分辨率的主要因素。
            可通过减小孔径半角$\alpha$减小球差,但会使衍射效应限制的分辨率变大,故关键在于确定最佳的孔径半角。
            \item 提高电磁透镜分辨率的主要途径是减小电子束波长,如提高加速电压,和减小球差系数。
        \end{enumerate}
        \item 透射电镜的组成
        \begin{enumerate}
            \item 电子光学系统
            \begin{enumerate}
                \item 照明系统
                \begin{enumerate}
                    \item 电子枪:分为热发射电子枪与场发射电子枪。热发射电子枪由阴极,栅极,阳极组成。栅极可控制阴极发射电子的有效区域,自偏压回路的作用是
                    稳定和调节束流。场发射电子枪性能优异,具有束斑尺寸小,亮度高,能量分散度小等特点
                    \item 聚光镜:高性能透射电镜采用双聚光镜系统。第一透光镜是强励磁透镜,作用是缩小或调节束斑尺寸。第二聚光镜是弱励磁透镜,以调节照明强度
                    。聚光镜的作用是以最小的损失,减小和调节束斑尺寸,调节照明强度和照明孔径半角。
                \end{enumerate}
                \item 成像系统
                \begin{enumerate}
                    \item 物镜:强励磁、短焦距,用来形成第一幅图像的透镜,在物镜背焦面上形成衍射花样,在像平面上形成显微图像,所以透射电镜分辨率的高低主要
                    取决于物镜,物镜是最核心的部件。物镜的分辨率主要取决于极靴形状和加工精度,极靴内孔和上下极靴之间的距离越小,物镜的分辨率越高。
                    \item 中间镜:弱励磁、长焦距、变倍率。可控制电镜的总放大倍数,可实现透射电镜成像操作与衍射操作的转换:将中间镜的物平面与物镜像平面重合
                    ,则为成像操作;将中间镜物平面与物镜背焦面重合,则为衍射操作
                    \item 投影镜:强励磁、短焦距。进一步放大中间镜的像
                \end{enumerate}
                \item 观察记录系统
            \end{enumerate}
            \item 电源与控制系统
            \item 真空系统
        \end{enumerate}
        \item 几种光阑的主要作用
        \begin{enumerate}
            \item 聚光镜光阑:限制和改变照明孔径半角,改变照明强度;安装在第二聚光镜下方
            \item 物镜光阑:减小物镜的球差,选择成像电子束以获得明场像或暗场像,可提高图像衬度;安装在物镜的背焦面上;也称为衬度光阑
            \item 选区光阑:衍射分析时,限制和选择样品分析区域,实现选区电子衍射;安放在物镜的像平面上;也称为视场光阑
        \end{enumerate}
    \end{enumerate}
    \section{192001材料表征文档考点}
    \begin{enumerate}
        \item 解释X射线的产生及X射线谱
        \item 解释特征X射线谱
        \item 解释X射线与物质相互作用的几种主要效应
        \item 解释X射线的散射
        \item 导出布拉格方程
        \item 六方晶系的指数变换
        \item 布拉格方程的应用
        \item 厄瓦尔德图解倒易空间的衍射方程
        \item X射线衍射的几种主要方法
        \item 几种基本点阵的消光规律
        \item X射线衍射仪的基本组成
        \item X射线定性分析基本原理
        \item X射线衍射PDF卡片的说明
        \item 内应力的分类、分布与衍射效应
        \\内应力指产生应力的各种因素不复存在时,由于形变、体积变化不均匀而残留在构件内部并自身保持平衡的应力
        \begin{enumerate}
            \item 第一类应力
            \begin{enumerate}
                \item 分类:指在物体宏观体积内存在并平衡的内应力。当其被释放后,物体的宏观体积或形状将会发生变化
                \item 分布:存在于各个晶粒的内应力在很多晶粒范围内的平均值,是较大体积宏观变形不协调的结果
                \item 衍射效应:又称为宏观应力或残余应力,使衍射线位移
            \end{enumerate}
            \item 第二类应力
            \begin{enumerate}
                \item 分类:指在数个晶粒范围内存在并平衡的内应力。这种平衡被破坏时也会出现尺寸变化
                \item 分布:在晶粒尺度范围内应力的平均值,为各个晶粒或晶粒区域之间变形不协调的结果
                \item 衍射效应:又称为微观应力,引起衍射线线型变化。第二类应力是一种十分重要的中间环节,通过它可将第一类应力与第三类应力联系起来,
                构成一个完整的应力系统
            \end{enumerate}
            \item 第三类内应力
            \begin{enumerate}
                \item 分类:指在若干个原子范围内存在并平衡的内应力,如各种晶体缺陷:空位、间隙原子、位错等。这种平衡被破坏时不会产生尺寸的变化
                \item 分布:晶粒内局部内应力相对第二类内应力值的波动,与晶体缺陷形成的应变场有关
                \item 衍射效应:又称为晶格畸变应力或超微观应力,使衍射强度降低
            \end{enumerate}
        \end{enumerate}
        \item 对电子显微分析的理解
        \item 对分辨率的理解
        \item 对电磁透镜像差的理解
        \item 电子显微镜的基本组成
        \item 几种光阑的主要作用
    \end{enumerate}

    

\end{document}