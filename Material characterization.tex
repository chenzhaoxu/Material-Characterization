%\documentclass[UTF8]{ctexart}
\documentclass[12pt,a4paper]{article}

\usepackage[UTF8]{ctex}
\usepackage[left=0.5in,right=0.5in,%
top=0.75in,bottom=0.75in]{geometry}
\usepackage{amsthm}

\title{\heiti 材料表征方法}
\author{\kaishu 岚岫}
\date{\today}

\begin{document}
    \maketitle
    %\tableofcontents
\begin{abstract}
    大三下学期与大四上学期陆续修读了几门关于材料表征方法的课程,此文档将陆续整理相关知识,可作为考试复习提纲和学习参考。
\end{abstract}
\section{X射线衍射分析}
\section{电子显微分析}
\section{同步辐射}
\section{中子科学}
\section{192001材料表征上课考点}
    \begin{enumerate}
        \item X射线是如何产生的?
        \begin{enumerate}
            \item 画出X射线管的结构
            \item 主要由阴极(W灯丝)和用Cu、Cr、Fe、Mo等纯金属制成的阳极靶组成
            \item 阴极通电加热,在阴阳两极之间加以直流高压(约数万伏)
            \item 阴极发射的大量电子高速飞向阳极,与阳极碰撞产生X射线
        \end{enumerate}
        \item X射线的一些基本性质
        \begin{enumerate}
            \item X射线是一种波长很短的电磁波
            \item X射线的波长范围为0.01$nm$~0.25$nm$
            \item X射线是一种横波,由交替变化的电场和磁场组成
            \item X射线具有波粒二象性,因其波长较短,其粒子性较为突出,即可以把X射线看成是一束具有一定能量的光量子流,$$E=h v=h c / \lambda$$
            \item X射线穿过不同介质时,折射系数接近于1,几乎不发生折射现象
            \item X射线肉眼不可见,但可使荧光物质发光、能使赵向地板感光、能使一些气体产生电离现象
            \item X射线的穿透能力强,能穿透对可见光不透明的材料,特别是波长在0.1$nm$以下的硬X射线
            \item X射线照射到晶体时,将产生散射、干涉、衍射等现象,与光线的绕射现象类似
            \item X射线具有破坏杀死生物组织细胞的作用
        \end{enumerate}
        \item 管电压、管电流、阳极靶原子序数对连续谱的作用
        \item 特征X射线谱的产生
        \begin{enumerate}
            \item 当X射线管压高于靶材相应的某一特征值$U_k$时,在某些特定波长位置上,将出现一系列强度很高、波长范围很窄的线状光谱,称为特征谱或标识谱。
            其波长与阳极靶材的原子序数有确定关系,故可作为靶材的标志和特征$$\sqrt{\frac{1}{\lambda}}=K_{2}(Z-\sigma)$$表明阳极靶材的原子序数越大,同一线系的特征谱波长越短。
            \item 冲向阳极的电子若具有足够能量,将内层电子击出而成为自由电子。此时,原子处于高能不稳定状态,必然自发的向稳态过渡。若$L$层电子跃迁到$K$层填补空位,原子
            由$K$激发态转为$L$激发态,能量差以X射线的形式释放。这就是特征X射线,称为$K_\alpha$射线。
            \item 由于$L$层内还有能量差别很小的亚能级,不同亚能级的电子跃迁将辐射$K_{\alpha1}$、$K_{\alpha2}$射线。
            \item $$\lambda_{K_{\alpha 1}}<\lambda_{K_{\alpha 2}}, \quad I_{K_{\alpha 1}} \approx 2 I_{K_{\alpha 2}}$$
        \end{enumerate}
        \item X射线的真吸收
        \begin{enumerate}
            \item 光电效应:当入射X射线光量子能量等于或略大于吸收体原子某壳层电子的结合能时,电子易获得能量从内层逸出,成为自由电子,成为光电子。
            这种光子击出电子的现象成为光电效应。光电效应将消耗大部分入射能量,导致吸收系数突增。
            \item 荧光效应:因光电效应处于相应的激发态的原子,将随之发生如前所述的外层电子向内层跃迁的过程,同时辐射出特征X射线,称X射线激发
            产生的特征辐射为二次特征辐射,称这种光致发光的现象为荧光效应。
            \item 俄歇效应:原子$K$层电子被击出后,$L$层一个电子跃入$K$层填补空位,而另一个$L$层电子获得能量溢出成为俄歇电子。称这种一个$K$层空位被两个$L$
            层空位取代的过程为俄歇效应。
            \item 荧光X射线与俄歇电子均为物质的化学成分信号。荧光X射线用于重元素的成分分析,俄歇电子用于表面轻元素分析。
            \item X射线穿过物质后强度会产生衰减。强度衰减主要是由于真吸收消耗于光电效应与热效应。强度衰减还有一小部分是由于X射线偏离了原来的入射方向,即散射。
        \end{enumerate}
        \item X射线的相干散射
        \begin{enumerate}
            \item 当入射X射线与受原子核束缚较紧的电子相遇,使电子在X射线在交变电场作用下发生受迫振动,向四周辐射与入射X射线波长相同的辐射
            \item 因各电子散射的X射线波长相同,有可能相互干涉,因此称相干散射,亦称经典散射
            \item 物质对X射线的散射可以认为只是电子的散射
            \item 相干散射仅占入射能量的极小部分
            \item 相干散射是X射线衍射分析的基础
        \end{enumerate}
        \item X射线的不相干散射:当X射线与自由电子或受核束缚较弱的电子碰撞时,使电子获得部分能量离开原子核而成为反冲电子,X射线能量损失,而发生波长变长的不相干散射。
        \item 
    \end{enumerate}
    \section{192001材料表征文档考点}
    \begin{enumerate}
        \item 解释X射线的产生及X射线谱
        \item 解释特征X射线谱
        \item 解释X射线与物质相互作用的几种主要效应
        \item 解释X射线的散射
        \item 导出布拉格方程
        \item 六方晶系的指数变换
        \item 布拉格方程的应用
        \item 厄瓦尔德图解倒易空间的衍射方程
        \item X射线衍射的几种主要方法
        \item 几种基本点阵的消光规律
        \item X射线衍射仪的基本组成
        \item X射线定性分析基本原理
        \item X射线衍射PDF卡片的说明
        \item 内应力的分类、分布与衍射效应
        \item 对电子显微分析的理解
        \item 对分辨率的理解
        \item 对电磁透镜像差的理解
        \item 电子显微镜的基本组成
        \item 几种光阑的主要作用
    \end{enumerate}

    

\end{document}