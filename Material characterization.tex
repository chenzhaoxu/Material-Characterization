%\documentclass[UTF8]{ctexart}
\documentclass[12pt,a4paper]{article}

\usepackage[UTF8]{ctex}
\usepackage[left=0.5in,right=0.5in,%
top=0.75in,bottom=0.75in]{geometry}
\usepackage{amsthm}
\usepackage{amsmath}

\title{\heiti 材料表征方法}
\author{\kaishu 岚岫}
\date{\today}

\begin{document}
    \maketitle
    %\tableofcontents
\begin{abstract}
    大三下学期与大四上学期陆续修读了几门关于材料表征方法的课程,此文档将陆续整理相关知识,可作为考试复习提纲和学习参考。
\end{abstract}
\section{X射线衍射分析}
\section{电子显微分析}
\section{同步辐射}
    \subsection{同步辐射源}
    \begin{enumerate}
        \item 同步辐射(synchrotron radiation):相对论性带电粒子在电磁场作用下沿弯转轨道行进时发出的电磁辐射,波长有一定范围,因同步辐射源而异,一般包含紫外光、红外光、可见光
            、X射线
        \item 同步辐射源主体:将带电粒子加速到并/或维持在相对论性状态的大中型粒子加速器。粒子可以为电子或正电子。加速器可分为环形加速器、电子直线加速器。
            环形加速器包括储存环与电子同步加速器,以储存环(store ring)为主。电子直线加速器可以为驱动的自由电子激光(free electron laser, FEL)装置。
        \item 加速度不为0时,运动电荷的电磁场可分为以下几种情况:
            \begin{enumerate}
                \item $a<0$, 韧致辐射;
                \item 粒子低速往复运动,$v$与$a$不断变号,产生震荡电荷的辐射;
                \item $v~c$, $a$与$v$大致垂直,产生同步辐射
                \item 近场(near field):只与速度有关,特征与匀速飞行的电荷的场相同,不产生辐射
                \item 远场(far field):坡印廷矢量的大小与加速度的平方成正比,从发光位置指向任意观察点,不断向远方辐射能量,导致非匀速直线运动的电荷能量产生损
                失。
            \end{enumerate}
        \item 同步辐射的特性:
            \begin{enumerate}
                \item 光谱:广阔:波长覆盖范围很宽,一般从红外线一直到X射线;连续与平滑:频谱既没有下凹的断点也没有凸起的特征峰;可调性:
                    使用单色器,可以从光束中选取一定波长与带宽的光
                \item 高度准直性:同步辐射功率基本上集中在电子弯转轨道的切线方向附近,单个电子在轨道上一点发出的同步辐射好像沿该方向伸出的细窄的光锥,对于单色光
                    光子能量越高,发散角越小。实际上,光束的水平张角由光束线的光阑开口决定。说明能流密度高
                \item 高辐射功率
                \item 高亮度(brilliance):即高的辐射能量集中程度。同步辐射光源用磁铁聚集结构约束电子束的横截面与发散角,与高辐射功率与高度准直性的优点相结合
                    使同步辐射光源的亮度远高于常规光源,能在很小的样品照射面积上、很小的空间角度内、很窄的能谱带宽区间提供足够多的单位时间光子数,获得很高的
                    位置分辨率、角度分辨率、光子能量分辨率。
                \item 偏振性:同步辐射源具有天然的偏振性。电矢量振动主要在与弯转轨道平面平行的方向上,偏振度依赖于光线与该平面的交角$\theta$,也是波长$\lambda$
                    的函数。将单个电子的同步辐射矢量分解,当$\theta$为0时,垂直分量不存在,为$100\%$的线偏振;当$\theta$不为0时,同步辐射有不同程度的椭圆偏振
                    ,旋转方向取决于观察点在平面的上方或下方。对于单色光,光子能量越高,平行分量比重越大,偏振度越高。辐射的偏振性对样品各向异性的实验至关重要
                \item 脉冲时间结构:电子因同步辐射损失的能量由高频加速电场补充。该电场强度随时间周期变化,必然将电子束分割为若干个不连续的束团。适宜于某些动态过程的研究。
                \item 高真空环境:光束不必通过隔窗与气体,受到的吸收与污染皆控制在最低限度内,对易被空气吸收的高能紫外线,即真空紫外光(vacuum ultraviolet, VUV)尤为可贵
                \item 可计算性:发光机制只涉及不受束缚的高能电子及其在磁场中的运动。无需考虑介质密度涨落、化学纯度、温度分布等难以测量的因素。光谱分布、偏振性、角分布等特性
                    都可以用公式计算
            \end{enumerate}
            \item 同步辐射源的分类:
                \begin{enumerate}
                    \item 一代光源:寄生运行,辐射是加速电子进行高能物理实验的副产品
                    \item 二代光源:粒子储存环的出现能让粒子束在环内以不变的能量,沿着固定的轨道稳定运行相当长一段时间。可实现双束高速连续对撞,使粒子反应的有效作用明显增大
                    \item 三代光源:插入元件的诞生可作为三代光源的出现标志。插入元件即在加速器中原来并未放置磁铁的直线节内安装的一种能产生空间交变磁场的元件
                    \item 四代光源:主要特征是高度相干的辐射,其亮度远高于三代,还可能有脉冲极短,平均功率高等特点
                \end{enumerate}
            \item 同步辐射的运行周期:注入束流、满极速(假设并非满能量注入)、稳定储存、废弃束流、再次注入
                \begin{enumerate}
                    \item 注入:将来自注入器的外来电子引入环中,使许多电子被环中的电磁场俘获
                    \item 慢加速:积累足够多的电子后全环磁场以很慢的速度(典型时间为几分钟)同步增强
                    \item 注入的几种分类:
                        \begin{enumerate}
                            \item 低能注入:衰减到某一程度之后放弃,然后进行磁铁标准化,即电流从0到极大值数次反复,以消除磁滞回线对电流-磁场关系的影响
                                再根据注入能量重设参数,从0注入,积累束流并慢加速,再从最大流强衰减
                            \item 满能量注入:机器参数无需变动,以补充束流代替丢弃束流,不必慢加速,若在衰减周期的中途加注一次,虽然最大值不变,但是流强
                                极小值明显增大,积分流强得益甚多
                            \item $Top-up$注入:将满能量注入推送到极致,注入频率很高但每次注入的电荷量不多,流强刚有衰减就及时补充,积分流强得到最大化
                                ,难点:注入的俘获效率必须很高,注入凸轨的匹配与同步必须相当完美
                        \end{enumerate}
                    \item 插入元件:在加速器中原来并未放置磁铁的直线节中安装的一种能产生空间交变磁场的元件,在最常见的平面型插入元件中上下相对的多个磁极
                        沿束流轨道周期性排列,磁场垂直于弯转平面,而符号和大小以类似三角函数的形式交替变化。在这种磁场中,束流必将以很小的振幅左右摇摆
                        ,蜿蜒前进。扭摆轨迹相对于无插入元件时直线的最大摆动角与同步辐射光锥半顶角的比值称为插入元件参数$K$。插入元件有:
                        \begin{enumerate}
                            \item 波荡器(undulator):$K<=4$,插入元件的峰值磁场不太强,而且磁场的周期较短,束流的摆动角相当小,电子在不同位置发出的光锥充分的重叠,
                                由于磁场的周期性而产生干涉作用。只有满足特定波长的光才会被加强,其他波长的光减弱。一般有足够多的周期数保证干涉的效果,使辐射的功率集中在
                                几条谱线附近,故其是亮度非常高的准单色光,亮度比普通的同步辐射功率高上千倍。全光束张角与单电子辐射天然张角相近。
                            \item 扭摆器(wiggler):$K>=10$,磁场教导,束流摆动角远大于同步辐射光锥半顶角,轨道扭摆幅度大。电子沿扭摆器全程发出的光扇形散开,基本上不发
                                光与光的干涉,只有功率的非相干叠加。辐射仍有连续平滑的频谱,与弯转磁铁产生的频谱相似。但因为磁场一般较强,往往有较高的特征光子能量,较
                                强的辐射功率,光子通量与有效磁极对数成正比。
                            \item 弯转磁铁(bending magnet):广阔、连续、平滑
                            \item 此处需补充若干图片
                        \end{enumerate}
                \end{enumerate}
            \item Important part:monochromator(单色器), mirrors to focus the beam and control the beam, beam defining slits(限束狭缝、限束光阑), 
                beam position monitors(BPM)(光束位置检测器), filters(take away unwanted radiation)(滤波器), vacuum pumps(真空泵)
            \item 光束线(beam line):对同步辐射光源产生的光束进行处理,如聚焦、单色化;将光束传输到实验装置的指定位置。Beam lines can select the 
                energy (wave length) and focus the beam on the sample, so different energies use different type of beam lines. Synchrotrons 
                generates energines from IR up to 500 keV, but above 150 keV the intensity is low.
            \item 前段区:是光束线与储存环的接口,具有真空保护、辐射屏蔽、光束准直及降低光束原件上热负载的功能
            \item 聚焦:由于物质对真空紫外及X射线的强烈吸收,在此能量波段,不能采用对可见光系统有效的透镜系统,通常采用基于几何原理的反射镜系统来进行光束的聚焦
                ,反射镜构成的聚焦系统由于像差等因素的影响,聚焦斑点最小尺寸通常大于几十微米。要想达到纳米尺寸的聚焦,目前的技术是采用基于干涉原理的波带片系统
            \item 单色:目前光的单色化技术都基于光的衍射原理,所使用的光栅间距需要与光的波长相当。X射线的波长~固体中原子间距~晶体单色器;真空紫外、软X射线~
                远大于固体中原子间距~人工刻画的光栅单色器
                \begin{enumerate}
                    \item hard x-rays: above 2.5keV, always use crystal monochromators
                    \item soft x-rays: below 1.5keV, always use grating monochromators
                    \item UV beam lines: between 1.5~2.5 eV is difficult
                \end{enumerate}
            \item 影响反射镜反射率的因素:
                \begin{enumerate}
                    \item 同步辐射诱发光学元件表面的光化学反应。即使在超高真空的条件下,也不可能完全排除$CO,CO_2$等含碳残留气体。故碳化合物对束线光学元件
                        的污染很难避免
                    \item 反射镜的表面粗糙度,反射镜可以作为低通滤波器滤去高能量光子,从而消除单色光的高次谐波输出
                \end{enumerate}
            \item 平面镜用于偏转光束;球面镜、柱面镜、椭球面镜、抛物面镜、超环面经等曲面镜可以对光束聚焦、放缩或偏转
            \item 球面镜的像差在掠入射时非常严重;非球面镜加工困难:将平面镜机械压弯,得到希望的柱面或抛物线柱面镜,压完后可以通过改变平面镜的厚度函数来调节
                压完后的曲面形状
            \item 超环面镜是可以达到完美聚焦的唯一选择。两种通过压弯机制达到超环面镜聚焦特性的方法:
                \begin{enumerate}
                    \item 两个压弯的平面镜分别实现子午线方向与弧矢方向的聚焦
                    \item 将柱面镜在子午线方向进行压弯
                \end{enumerate}
            \item Format定理:所有从A点传播到B点的光线总是走过相同的光程,而且此光程是所有AB间可能光程的极值。
            \item 面形误差的来源:加工与安装;热负载使镜面温度升高
            \item 可通过多面反射镜组合的方法来减小像差,但多面反射镜的组合有可能减少光强,一个比较折中的方案是双反射镜系统,典型双反射镜系统有:
                Kirkpatrick-Baes双反射镜系统、Namioka双球面镜系统
            \item 波带片可将光斑聚焦到纳米量级,通常来讲反射镜只能将光斑聚焦到微米量级。博导片的效应是基于光的波动性引起的衍射效应,由交错的透明的和不透明的
                圆环组成
            \item 光栅单色器根据常用的截面形状可以分为以下几类:补充图片
                \begin{enumerate}
                    \item 层状光栅(lamellar)
                    \item 闪耀光栅(blazed):更高的效率
                    \item 正弦光栅(sinusoidal)
                \end{enumerate}
            \item 光栅面可以是平面、球面与非球曲面。曲面光栅可以同时完成单色与聚焦的功能。从而省去聚焦反射镜以减小光强损失。由于曲面光栅可以同时完成两项功能,
                其可以覆盖的能量范围通常较窄。
            \item 真空紫外光束线的性能指标:光强、能量分辨率
            \item 晶体单色器:光的衍射效应
            \item 平晶单色器三种典型配置:补充图片
            \item 平晶单色器:能量误差与入射光束的角度成正比,故想要得到比较高的分辨率,入射光束必须有较好的准直性。对于发散的光束,可以使用入射狭缝遮蔽大发散角
                X射线。但这样意味着光强的损失,可以用弯晶单色器来解决上述困境
            \item Why double crystal mono?
                \begin{enumerate}
                    \item Better resolution
                    \item Beam at the same place all the time
                    \item Use Bragg law
                \end{enumerate}
    \end{enumerate}
\subsection{同步辐射表征方法}
    \begin{enumerate}
            \item Why high energy x-rays for material science?
                \begin{enumerate}
                    \item High penetration depth:
                    \begin{enumerate}
                        \item Non-destructively bulk properties measurable
                        \item Deeply buried structure accessible
                        \item Easy to design special-purpose in-situ equipment(growth, catalysis, strain\dots)
                    \end{enumerate}
                    \item Large Ewald sphere
                    \begin{enumerate}
                        \item Lines and planes in reciprocal space can be probed
                        \item Small Bragg angles (5~15), complete diffraction rings @ area detectors
                        \item Simultaneous WAXS and SAXA possible (广角X射线散射,小角X射线散射)
                    \end{enumerate}
                    \item Focusing to spot sizes in nm range possible (low emittance)
                    \begin{enumerate}
                        \item High spatial resolution narrowing the gap to electron microscopy
                        \item Combination of high penetration depth and high flux
                        \item Very shory data acquisition times possible
                    \end{enumerate}
                    \item High intensity (brilliance)
                    \begin{enumerate}
                        \item Time-resolved data: 2D area detector read out at 0.1~100Hz
                        \item Non-.....................of dynamic processes
                    \end{enumerate}
                \end{enumerate}
        \item What happens when a photon hits a surface?
            \begin{enumerate}
               \item Photons interact with electron
               \item Depends on the binding energy
               \item 补充图片 
            \end{enumerate}
        \item 同步辐射的频谱覆盖了从远红外直至$\gamma$射线的电磁波段,对于(软)X射线部分,与物质的相互作用包括散射、折射、吸收、荧光等
        \item 单个电子对(软)X射线的散射是非常弱的,在距离该电子几厘米处光强已衰弱$10_{-26}$,故只有散射相干叠加为衍射才可以测量
        \item (软)X射线波段,介质的折射率均略小于1,故为光疏介质。略小于1,十分趋近于1,故介质对(软)X射线的折射不很明显,只有当入射角~90时,即掠入射时
            ,才会发生明显的折射。对于光疏介质来说,从真空或空气入射到介质表面可能发生全反射。该原理可用于设计(软)X射线波段的反射光学元件,如反射镜、聚光镜等
        \item (软)X射线的吸收包括
    \end{enumerate}
\section{中子科学}
\section{192001材料表征上课考点}
    \begin{enumerate}
        \item X射线是如何产生的?
        \begin{enumerate}
            \item 画出X射线管的结构
            \item 主要由阴极(W灯丝)和用Cu、Cr、Fe、Mo等纯金属制成的阳极靶组成
            \item 阴极通电加热,在阴阳两极之间加以直流高压(约数万伏)
            \item 阴极发射的大量电子高速飞向阳极,与阳极碰撞产生X射线
        \end{enumerate}
        \item X射线的一些基本性质
        \begin{enumerate}
            \item X射线是一种波长很短的电磁波
            \item X射线的波长范围为0.01$nm$~0.25$nm$
            \item X射线是一种横波,由交替变化的电场和磁场组成
            \item X射线具有波粒二象性,因其波长较短,其粒子性较为突出,即可以把X射线看成是一束具有一定能量的光量子流,$$E=h v=h c / \lambda$$
            \item X射线穿过不同介质时,折射系数接近于1,几乎不发生折射现象
            \item X射线肉眼不可见,但可使荧光物质发光、能使赵向地板感光、能使一些气体产生电离现象
            \item X射线的穿透能力强,能穿透对可见光不透明的材料,特别是波长在0.1$nm$以下的硬X射线
            \item X射线照射到晶体时,将产生散射、干涉、衍射等现象,与光线的绕射现象类似
            \item X射线具有破坏杀死生物组织细胞的作用
        \end{enumerate}
        \item 管电压、管电流、阳极靶原子序数对连续谱的作用
            \begin{enumerate}
                \item 连续X射线谱的特点是:X射线的波长存在最小值$\lambda_{SWL}$,其强度在$\lambda_m$处有最大值
                \item 当管电压$U$升高时,各波长X射线的强度均提高,短波限$\lambda_{SWL}$,和强度最大值对应的波长$\lambda_m$减小
                \item 当管电流i增大时,各波长X射线的强度均提高,但是$\lambda_m$与$\lambda_{SWL}$不变
                \item 随阳极靶材的原子序数增大,连续X射线谱的强度提高,但$\lambda_m$与$\lambda_{SWL}$不变
            \end{enumerate}
        \item 特征X射线谱的产生
        \begin{enumerate}
            \item 当X射线管压高于靶材相应的某一特征值$U_k$时,在某些特定波长位置上,将出现一系列强度很高、波长范围很窄的线状光谱,称为特征谱或标识谱。
            其波长与阳极靶材的原子序数有确定关系,故可作为靶材的标志和特征$$\sqrt{\frac{1}{\lambda}}=K_{2}(Z-\sigma)$$表明阳极靶材的原子序数越大,同一线系的特征谱波长越短。
            \item 冲向阳极的电子若具有足够能量,将内层电子击出而成为自由电子。此时,原子处于高能不稳定状态,必然自发的向稳态过渡。若$L$层电子跃迁到$K$层填补空位,原子
            由$K$激发态转为$L$激发态,能量差以X射线的形式释放。这就是特征X射线,称为$K_\alpha$射线。
            \item 由于$L$层内还有能量差别很小的亚能级,不同亚能级的电子跃迁将辐射$K_{\alpha1}$、$K_{\alpha2}$射线。
            \item $$\lambda_{K_{\alpha 1}}<\lambda_{K_{\alpha 2}}, \quad I_{K_{\alpha 1}} \approx 2 I_{K_{\alpha 2}}$$
        \end{enumerate}
        \item X射线的真吸收
        \begin{enumerate}
            \item 光电效应:当入射X射线光量子能量等于或略大于吸收体原子某壳层电子的结合能时,电子易获得能量从内层逸出,成为自由电子,成为光电子。
                这种光子击出电子的现象成为光电效应。光电效应将消耗大部分入射能量,导致吸收系数突增。
            \item 荧光效应:因光电效应处于相应的激发态的原子,将随之发生如前所述的外层电子向内层跃迁的过程,同时辐射出特征X射线,称X射线激发
                产生的特征辐射为二次特征辐射,称这种光致发光的现象为荧光效应。
            \item 俄歇效应:原子$K$层电子被击出后,$L$层一个电子跃入$K$层填补空位,而另一个$L$层电子获得能量溢出成为俄歇电子。称这种一个$K$层空位被两个$L$
                层空位取代的过程为俄歇效应。
            \item 荧光X射线与俄歇电子均为物质的化学成分信号。荧光X射线用于重元素的成分分析,俄歇电子用于表面轻元素分析。
            \item X射线穿过物质后强度会产生衰减。强度衰减主要是由于真吸收消耗于光电效应与热效应。强度衰减还有一小部分是由于X射线偏离了原来的入射方向,即散射。
        \end{enumerate}
        \item X射线的相干散射
        \begin{enumerate}
            \item 当入射X射线与受原子核束缚较紧的电子相遇,使电子在X射线在交变电场作用下发生受迫振动,向四周辐射与入射X射线波长相同的辐射
            \item 因各电子散射的X射线波长相同,有可能相互干涉,因此称相干散射,亦称经典散射
            \item 物质对X射线的散射可以认为只是电子的散射
            \item 相干散射仅占入射能量的极小部分
            \item 相干散射是X射线衍射分析的基础
        \end{enumerate}
        \item X射线的不相干散射:当X射线与自由电子或受核束缚较弱的电子碰撞时,使电子获得部分能量离开原子核而成为反冲电子,X射线能量损失,而发生波长变长的不相干散射。
        \item 六方晶系的指数换算:$$\begin{array}{c}{U=u-t, V=v-t, W=w} \\ {u=(2 U-V) / 3} \\ {v=(2 V-U) / 3} \\ {t=-(u+v)} \\ {w=W}\end{array}$$
        \item 十四种布拉菲点阵
        \item 简单点阵的面间距公式:
        \\正交晶系:$$d_{h k l}=\frac{1}{\sqrt{h^{2} / a^{2}+k^{2} / b^{2}+l^{2} / c^{2}}}$$
        \\正方晶系:$$d_{h k l}=\frac{1}{\sqrt{\left(h^{2}+k^{2}\right) / a^{2}+l^{2} / c^{2}}}$$
        \\立方晶系:$$d_{h k l}=\frac{a}{\sqrt{h^{2}+k^{2}+l^{2}}}$$
        \\六方晶系:$$d_{h k l}=\frac{1}{\sqrt{\frac{4}{3}\left(h^{2}+h k+k^{2}\right) / a^{2}+l^{2} / c^{2}}}$$
        \item 倒易点阵的性质
        \item 非晶态物质结构的主要特征
        \item 关于布拉格方程的零散知识点:
            \begin{enumerate}
                \item X射线与原子内受束缚较紧的电子相遇时产生的相干散射波,在某些方向相互加强,而在某些方相互减弱,称这种散射波干涉的总结果为衍射
                \item X射线学以X射线在晶体中的衍射现象作为基础,衍射可以归结为衍射方向和衍射强度两方面的问题
                \item 衍射方向可以用劳厄方程或布拉格方程导出
                \item 劳厄方程在本质上解决了X射线衍射方向的问题,但难以直观的表达三维空间的衍射方向
                \item 布拉格定律将晶体的衍射看成是晶面族在特定方向对X射线的反射,非常简单方便
            \end{enumerate}
        \item 布拉格方程的导出
        \item 布拉格方程的应用:
        \begin{enumerate}
            \item 布拉格方程是X射线衍射分析中最重要的基础公式,能简单方便的说明衍射基本关系
            \item 用已知波长$\lambda$的X射线照射晶体,通过衍射角$2\theta$的测量计算晶体中各晶面的面间距$d$,这就是X射线结构分析
            \item 用已知面间距$d$的晶体反射样品激发的X射线,通过衍射角$2\theta$的测量计算X射线的波长$\lambda$,这就是X射线光谱分析 
        \end{enumerate}
        \item 几种点阵的结构因子计算结果
        \begin{enumerate}
            \item 简单点阵能产生衍射的干涉面指数$(HKL)$的平方和之比为:1:2:3:4:5:\dots
            \item 体心点阵
            \begin{enumerate}
                \item 当$H+K+L$为奇数时,$\left|\boldsymbol{F}_{H K L}\right|^{2}=\mathbf{0}$,衍射强度为零
                \item 当$H+K+L$为偶数时,$\left|\boldsymbol{F}_{\boldsymbol{H K L}}\right|^{2}=4 f^{2}$,晶面能产生衍射,这些干涉面的指数平方和之比
                    为2:4:6:8:\dots
            \end{enumerate}
            \item 面心点阵
            \begin{enumerate}
                \item 当$H,K,L$奇偶混合时,$\left|\boldsymbol{F}_{H K L}\right|^{2}=\mathbf{0}$,衍射强度为零
                \item 当$H,K,L$全奇全偶时,$\left|\boldsymbol{F}_{\boldsymbol{H K L}}\right|^{2}=16 f^{2}$,晶面能产生衍射,这些干涉面的指数平方和之比
                    为3:4:8:11:12:\dots
            \end{enumerate}
            \item 消光规律
            \begin{enumerate}
                \item 简单立方:$h,k,l$为任意整数时,均无消光现象
                \item 面心立方:$h,k,l$为异性数时,会产生消光,如\{110\}\{100\}\{210\}等面族
                \item 体心立方:$h+k+l$为奇数时,会产生消光,如\{100\}\{111\}\{221\}等面族
                \item 密排六方:$h+2k=3n$,且$l$为奇数时,会产生消光,如\{001\}\{111\}\{221\}等面族
            \end{enumerate}
            \item 固溶体出现有序化后,使无序固溶体因消光而失去的衍射线重新出现
        \end{enumerate}
        \item 多重因子:某种晶面的等同晶面数增加,参与衍射的几率随之增大,相应衍射强度也随之增加。晶面的等同晶面数对衍射强度的影响,称为多重因子$P$,
            多重因子与晶体的对称性及晶面指数有关。各晶系晶面族的多重因子可见下表:
        \item 非晶态物质结构的主要特征
        \item 识别PDF
        \item 光学显微镜的分辨率:$\Delta r_{0} \approx \frac{1}{2} \lambda$
        \item 在200$kV$的加速电压下,电子波的波长为0.00251$nm$,随加速电压上升,电子波波长下降,分辨率上升
        \item 电磁透镜的像差
        \begin{enumerate}
            \item 几何像差:又称为单色光引起的像差
            \begin{enumerate}
                \item 球差:由于透镜中心区域和边缘区域对电子的折射能力不同形成的。球差大小为$$\Delta r_{s}=\frac{1}{4} C_{s} \alpha^{3}$$
                    公式中$C_{S}$是球差系数,$\alpha$为孔径半角,从公式可以看出,减小球差的途径是减小球差系数与小孔径角成像。
                \item 像散:由于透镜磁场非旋转对称性引起不同方向的聚焦能力出现差别。像散大小为$$\Delta r_{A}=\Delta f_{A} \alpha$$式中$\Delta f_{A}$
                    为磁场出现非旋转对称时的焦距差,故可通过引入强度和方位均可调节的矫正磁场消除像散
            \end{enumerate}
            \item 色散:波长不同的多色光引起的像差。是透镜对能量不同电子的聚焦能力的差别引起的。色差大小为
                $\boldsymbol{\Delta} \boldsymbol{r}_{c}=\boldsymbol{C}_{c} \alpha\left|\frac{\boldsymbol{\Delta} \boldsymbol{E}}{\boldsymbol{E}}\right|$
                式中$C_{s}$是色差系数,$\Delta E / E$是电子能量变化率,可通过稳定加速电压和单色器来减小色差
            \item 球差系数与色差系数是电磁透镜的指标之一,其大小除了与透镜结构、极靴形状和加工精度等有关外,还受激磁电流的影响。球差系数与色差系数均随透镜激磁电
                流的增大而减小。故若要减少电磁透镜的像差,透镜线圈应尽可能通以大的激磁电流
        \end{enumerate}
        \item 分辨率:电磁透镜的分辨率由衍射效应和球面像差决定
        \begin{enumerate}
            \item 衍射效应对分辨率的影响:$$\Delta r_{0}=\frac{0.61 \lambda}{N \sin \alpha}$$
                式中$\lambda$是波长,$N$是介质的相对折射率,$\alpha$是透镜的孔径半角,
                波长$\lambda$越小,孔径半角$\alpha$越大,衍射效应限定的分辨率半径就越小,透镜的分辨率就越高
            \item 由球差、像散、色差所限定的分辨率分别为$\Delta r_{S}$、$\Delta r_{A}$、$\Delta r_{C}$,其中球差是限制透镜分辨率的主要因素。
                可通过减小孔径半角$\alpha$减小球差,但会使衍射效应限制的分辨率变大,故关键在于确定最佳的孔径半角。
            \item 提高电磁透镜分辨率的主要途径是减小电子束波长,如提高加速电压,和减小球差系数。
        \end{enumerate}
        \item 透射电镜的组成
        \begin{enumerate}
            \item 电子光学系统
            \begin{enumerate}
                \item 照明系统
                \begin{enumerate}
                    \item 电子枪:分为热发射电子枪与场发射电子枪。热发射电子枪由阴极,栅极,阳极组成。栅极可控制阴极发射电子的有效区域,自偏压回路的作用是
                        稳定和调节束流。场发射电子枪性能优异,具有束斑尺寸小,亮度高,能量分散度小等特点
                    \item 聚光镜:高性能透射电镜采用双聚光镜系统。第一透光镜是强励磁透镜,作用是缩小或调节束斑尺寸。第二聚光镜是弱励磁透镜,以调节照明强度
                        。聚光镜的作用是以最小的损失,减小和调节束斑尺寸,调节照明强度和照明孔径半角。
                \end{enumerate}
                \item 成像系统
                \begin{enumerate}
                    \item 物镜:强励磁、短焦距,用来形成第一幅图像的透镜,在物镜背焦面上形成衍射花样,在像平面上形成显微图像,所以透射电镜分辨率的高低主要
                        取决于物镜,物镜是最核心的部件。物镜的分辨率主要取决于极靴形状和加工精度,极靴内孔和上下极靴之间的距离越小,物镜的分辨率越高。
                    \item 中间镜:弱励磁、长焦距、变倍率。可控制电镜的总放大倍数,可实现透射电镜成像操作与衍射操作的转换:将中间镜的物平面与物镜像平面重合
                        ,则为成像操作;将中间镜物平面与物镜背焦面重合,则为衍射操作
                    \item 投影镜:强励磁、短焦距。进一步放大中间镜的像
                \end{enumerate}
                \item 观察记录系统
            \end{enumerate}
            \item 电源与控制系统
            \item 真空系统
        \end{enumerate}
        \item 几种光阑的主要作用
        \begin{enumerate}
            \item 聚光镜光阑:限制和改变照明孔径半角,改变照明强度;安装在第二聚光镜下方
            \item 物镜光阑:减小物镜的球差,选择成像电子束以获得明场像或暗场像,可提高图像衬度;安装在物镜的背焦面上;也称为衬度光阑
            \item 选区光阑:衍射分析时,限制和选择样品分析区域,实现选区电子衍射;安放在物镜的像平面上;也称为视场光阑
        \end{enumerate}
        \item 晶带轴的计算\\正点阵中同时平行于某一晶向$[uvw]$的所有晶面构成一个晶带,这个晶向称为晶带轴。通过倒易原点(000)的倒易平面称为零层倒易面,因为
            $r=[uvw]$与零层倒易面$(uvw)_{0}^{*}$垂直,所以位于$(uvw)_{0}^{*}$上的倒易矢量$g_{hkl}$也与$r$垂直,故有$g_{h k l} \cdot r=0$
            ,即$h u+k v+l w=0$,上式即为晶带轴定理。晶带定理给出了晶面指数$(hkl)$和晶带轴指数$[uvw]$之间的关系。用晶带定理可求解已知两晶面的交线
            ,即晶带轴指数。例如:已知两个晶面指数分别为$\left(h_{1} k_{1} l_{1}\right)$、$\left(h_{2} k_{2} l_{2}\right)$,带入晶带轴定理有:
            $$\begin{array}{l}{h_{1} u+k_{1} v+l_{1} w=0} \\ {h_{2} u+k_{2} v+l_{2} w=0}\end{array}$$,解此方程组有:
            $$\left\{\begin{array}{l}{{u}=k_{1} l_{2}-k_{2} l_{1}} \\ {{v}=l_{1} h_{2}-l_{2} h_{1}} \\ {w={h}_{1} k_{
            2}-{h}_{2} {k}_{1}}\end{array}\right.$$   \\\emph{图片}
        \item 衍射斑点的存在与标定
        \item 若衍射斑点不对称,则倾转样品使其对称
        \item 晶体结构衍射花样的标定
        \item 电子衍射的基本公式:样品安放在反射球心$O$处,在其下方距离$L$处时荧光屏或底片,$O^{'}$是透射斑点,$G^{'}$是衍射斑点,因为$2\theta$
            很小,$g_{hkl}$与$k$接近垂直,故可得$\Delta O O^{*} G \sim \Delta O O^{\prime} G^{\prime}$,所以有$R / L=g / k$
            ,即$R d=L \lambda$。上式即为电子衍射的基本公式。式中,$L$是相机长度,$\lambda$是电子束波长,$d$是衍射晶面间距。$K=L\lambda$
            称为电子衍射相机常数。
            \\ \emph{图片}
        \item 选区电子衍射:入射电子束穿过样品后,在物镜背焦面上形成衍射花样,在物镜像平面上形成图像。若在物镜像平面处加一光阑,只允许$A^{'}B^{'}$范围内
            的电子通过,而挡住$A^{'}B^{'}$范围以外的电子,最终到达荧光屏形成衍射花样的电子只来自于样品的$AB$区域。此光阑起到的作用起到了限制和选择形成最终
            衍射花样的样品区域的作用。利用选区电子衍射可在多晶样品中获得单晶衍射花样,可实现组织形貌观察和晶体结构分析的微区对应。
            \\\emph{图片}
        \item 衍射斑点缩小,周期加大?????
        \item 相位衬度(高分辨电子显微镜成像原理)
            
        \item 透射电镜最重要的三部分:样品、物镜、电子束
        \item  \em 电子衍射与X射线的比较:
        \begin{enumerate}
            \item 电子波波长$\lambda$很小,故衍射角$2\theta$很小,约为0.01弧度、发射球半径$1/\lambda$很大,在倒易原点附近的反射面接近平面
            \item 透射电镜的样品厚度$t$很小,导致倒易点阵扩展量$1/t$很大,使略偏于布拉格条件的晶面也能发生衍射
            \item 当晶带轴$[uvw]$与入射线平行时,在与反射球面相切的零层倒易面上,倒易原点附近的阵点均能与反射球相截,从而产生衍射,所以单晶衍射花样是二维倒易平面的投影
            \item 原子对电子的散射因子比对X射线的散射因子约大4个数量级,故电子衍射强度较高,适用于微区结构分析,且拍摄衍射花样所需时间很短
        \end{enumerate}
                \em
        \item 
    \end{enumerate}
    \section{192001材料表征文档考点}
    \begin{enumerate}
        \item 解释X射线的产生及X射线谱
        \item 解释特征X射线谱
        \item 解释X射线与物质相互作用的几种主要效应
        \item 解释X射线的散射
        \item 导出布拉格方程
        \item 六方晶系的指数变换
        \item 布拉格方程的应用
        \item 厄瓦尔德图解倒易空间的衍射方程
        \item X射线衍射的几种主要方法
        \begin{enumerate}
            \item 劳厄法:劳厄法是最早的X射线方法,采用连续X射线照射不动的单晶体,用垂直于入射线的平底板记录衍射线而得到劳厄斑点,连续谱的波长范围为
                $\lambda_{0} \sim \lambda_{m}$,其中波长满足布拉格条件晶面将发生衍射,主要用于单晶取向测定及晶体对称性研究。
            \item 周转晶体法:周转晶体法采用单色X射线照射转动的单晶体,并用以晶体旋转轴为轴线的圆筒板记录衍射花样。晶体转动时,某晶面与X射线间夹角$\theta$
                将连续变化,而在某些特定位置满足布拉格方程条件而产生衍射斑点,衍射花样呈层线分布。主要用于单晶取向测定及对称性研究。
            \item 粉末法:粉末法用单晶X射线照射多晶试样。粉末法是X射线中最常用的方法,可以用粉末试样或块状样品,,其衍射花样能提供多种信息。可用于
                晶体结构测定、物相定性和定量分析、精确测定点阵常数以及材料内应力、织构、晶粒尺寸等。粉末法是各种多晶体X射线分析的总称。其中德拜-谢乐最具典型性
                ,目前最实用的方法是X射线衍射仪法。
        \end{enumerate}
        \item 几种基本点阵的消光规律
        \item X射线衍射仪的基本组成:20世纪50年代以前,X射线衍射分析基本上是利用底片记录衍射花样,即各种照相技术。目前,X射线衍射仪已经基本取代了照相法
            ,广泛应用于诸多研究领域。衍射仪具有测量方便、快速、准确等优点,它与计算机结合,使其操作、数据测量和处理大体上实现了自动化。X射线衍射仪主要由
            X射线发生器、测角仪、辐射探测器、记录单元和自动控制单元组成,其中测角仪是仪器的核心部件。
            \begin{enumerate}
                \item X射线测角仪:平板试样$D$安装在可绕轴$O$旋转的试样台$H$上,$S$处发射的一束发散X射线照射到试样上时,满足布拉格条件的晶面,其反射
                    线形成一收敛光束,计数管$C$连同狭缝$F$随支架$E$绕$O$旋转,在适当位置接收反射线。测角仪保持试样-计数管转动,即样品转过$\theta$
                    ,计数管恒转过$2\theta$。当试样和计数管连续转动时,衍射仪将自动绘出衍射强度随$2\theta$的变化曲线 \\\emph{图片}
                \begin{enumerate}
                    \item 试样:粉末试样压在样品框内,其粒度约为微米至几十微米,过粗时衍射强度不稳定,过细时衍射线宽化,也可采用块状样品,照磨面需磨平浸蚀。
                    \item 光学布置:$S$为线焦点,$K$为发散狭缝,$L$为防散射狭缝,$F$为接受狭缝,作用是限制射线的水平发散度。$S_1,S_2$为梭拉狭缝,用以限制射
                        线在竖直方向的发散度  \\\emph{图片}
                    \item 衍射几何:发散的入射线和平板试样的相对位置,使衍射线刚好在测角仪圆周上收敛。为使聚焦良好的X射线进入计数管,要求X射线管焦斑$S$、试样被照射表面
                        $MON$、衍射线汇聚点$F$,必须位于同一聚焦圆上。聚焦圆直径随$\theta$变而改变,$\theta$较小时其直径较大。工作时试样和探测器保持
                        $\theta-2\theta$联动,在X射线照射的大量晶粒中,只有平行于试样表面的晶面$(HKL)$才可能发生衍射 \\\emph{图片}
                    \item 弯晶单色器:
                \end{enumerate}
                \item 探测与记录系统
                \begin{enumerate}
                    \item 探测器
                    \begin{enumerate}
                        \item 正比计数器:反应快、能量分辨率高、背景脉冲低、计数率高、性能稳定;但对温度比较敏感,电压稳定度要求高
                        \item 闪烁计数器:分辨时间短,计数效率高;缺点是背底脉冲热噪声较高,晶体易受潮而失效
                    \end{enumerate}
                    \item 计数测量的主要电路:
                    \begin{enumerate}
                        \item 脉冲高度分析器
                        \item 定标器
                        \item 计数率计
                    \end{enumerate}
                \end{enumerate}
            \end{enumerate}
        \item X射线定性分析基本原理:
            \begin{enumerate}
                \item X射线衍射分析以晶体结构为基础,每种结晶物质都有其特定的结构参数,包括点阵类型、单胞中原子种类,数目和位置及单胞大小等。
                \item 这些结构参数在X射线衍射花样中必有所反映。多晶体物质衍射线条的数目、位置以及强度,是该种物质的特征,因而可以成为鉴别物相的标志。
                \item 世界上不存在衍射花样完全相同的两种物质,因此可以利用衍射花样与标准物质衍射卡片对照进行物相鉴定。
                \item 衍射线条的位置由$2\theta$决定,而$\theta$取决于波长$\lambda$及晶面间距$d$,其中$d$是晶体结构决定的基本量。应用时,将待测
                    花样和标准花样$d$及$I$系列对照,即可确定物相。
            \end{enumerate}
        \item X射线衍射PDF卡片的说明 \\\emph{图片}
            \begin{enumerate}
                \item 第一栏为物质的化学式和英文名称
                \item 第二栏为获得衍射数据的实验条件
                \item 第三栏为物质的晶体学数据
                \item 第四栏为样品来源、制备和化学分析等数据,还有获得数据的温度以及卡片的替换说明等
                \item 第五栏为物质的面间距、衍射强度以及对应的晶面指数
                \item 第六栏为卡片号
                \item 第七栏为卡片的质量标记
            \end{enumerate}
        \item 内应力的分类、分布与衍射效应
        \\内应力指产生应力的各种因素不复存在时,由于形变、体积变化不均匀而残留在构件内部并自身保持平衡的应力
        \begin{enumerate}
            \item 第一类应力
            \begin{enumerate}
                \item 分类:指在物体宏观体积内存在并平衡的内应力。当其被释放后,物体的宏观体积或形状将会发生变化
                \item 分布:存在于各个晶粒的内应力在很多晶粒范围内的平均值,是较大体积宏观变形不协调的结果
                \item 衍射效应:又称为宏观应力或残余应力,使衍射线位移
            \end{enumerate}
            \item 第二类应力
            \begin{enumerate}
                \item 分类:指在数个晶粒范围内存在并平衡的内应力。这种平衡被破坏时也会出现尺寸变化
                \item 分布:在晶粒尺度范围内应力的平均值,为各个晶粒或晶粒区域之间变形不协调的结果
                \item 衍射效应:又称为微观应力,引起衍射线线型变化。第二类应力是一种十分重要的中间环节,通过它可将第一类应力与第三类应力联系起来,
                构成一个完整的应力系统
            \end{enumerate}
            \item 第三类内应力
            \begin{enumerate}
                \item 分类:指在若干个原子范围内存在并平衡的内应力,如各种晶体缺陷:空位、间隙原子、位错等。这种平衡被破坏时不会产生尺寸的变化
                \item 分布:晶粒内局部内应力相对第二类内应力值的波动,与晶体缺陷形成的应变场有关
                \item 衍射效应:又称为晶格畸变应力或超微观应力,使衍射强度降低
            \end{enumerate}
        \end{enumerate}
        \item 对电子显微分析的理解:
            \begin{enumerate}
                \item 利用电子显微镜观察和分析材料的组织结构,称为电子显微分析术
                \item 电子显微镜是以电子束为光源的显微分析仪器,主要包括:透射电子显微镜、扫描电子显微镜和电子探针
                \item 电子显微镜的分辨率很高,目前透射电子显微镜的分辨率已优于0.1$nm$,达到了原子尺度
                \item 电子显微镜的分析功能很多,目前一台电子显微镜可兼有微观组织形貌观察、晶体结构、微区成分等多种分析功能
                \item 第一台电子显微镜于20世纪30年代问世,经历了几个阶段的发展,使电子显微分析技术已成为材料科学等研究领域中最重要的分析手段之一
            \end{enumerate}
        \item 对分辨率的理解
        \item 对电磁透镜像差的理解
        \item 电子显微镜的基本组成
        \item 几种光阑的主要作用
    \end{enumerate}

    

\end{document}